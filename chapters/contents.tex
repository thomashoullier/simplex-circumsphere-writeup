\section{Notations}
We will refer to:
\begin{itemize}
\item $n$ the dimensionality
\item A simplex with $n+1$ vertices
\item The vertices $V_i$, $i\in \llbracket 0, n \rrbracket$
\item The coordinates of vertex $V_i$: $V_i^j$, $j\in\llbracket 1, n \rrbracket$
\item The circumsphere with center $C$, with coordinates $C^j$, and radius $R$
\end{itemize}

\section{Method 1: Circumsphere equation}
This method is taken from a blog post by Gautam Manohar \cite{Gautam:2019}.
Any vertex $V_i$ is on the circumsphere (\cref{eq:vertex-on-sphere}),

\begin{equation} \label{eq:vertex-on-sphere}
\forall i, \sum_j \left( V_i^j - C^j \right)^2 = R^2
\end{equation}

Arbitrarily, we take the first equation, for the first vertex $V_{0}$, and
subtract it from all the others. After some expansion and cancellation of terms,
we obtain \cref{eq:row-equations}.

\begin{equation} \label{eq:row-equations}
\forall i \in \llbracket 1, n \rrbracket,
2 \sum_j \left( V_0^j - V_i^j \right) \cdot C^j =
  \sum_j \left( V_0^j \right)^2 - \sum_j \left( V_i^j \right)^2
\end{equation}

This is readily expressed in matrix format, with $i$ indexing the rows
in a linear system of equations (\cref{eq:matrix-system}).

\begin{equation} \label{eq:matrix-system}
A \cdot x = b
\end{equation}

With:
\begin{itemize}
\item $A$ a square matrix with elements $A_{i,j}$, of size $n$ (\cref{eq:a})
\item $x$ the vector of $C^j$ coordinates, of length $n$.
\item $b$ a right hand side vector defined by \cref{eq:b}, of length $n$.
\end{itemize}

\begin{equation} \label{eq:a}
A_{i,j} = 2 \left( V_0^j - V_i^j \right)
\end{equation}

\begin{equation} \label{eq:b}
b_{i} = \sum_j \left( V_0^j \right)^2 - \sum_j \left( V_i^j \right)^2
\end{equation}

We can solve for $x = C$ using any linear equation solver. Next, we can
retrieve $R$ from the distance of any vertex from the center, for example
eq (\cref{eq:radius}).

\begin{equation} \label{eq:radius}
R = \sqrt{\sum_j \left( V_0^j - C^j \right)^2}
\end{equation}

Thus, we have computed the center $C$ of the circumsphere, and $R$ its radius.

\section{Method 2: Cayley-Menger matrix}
As noted by Westendorp \cite{Westendorp:2013}, building on a work by Coxeter
\cite{Coxeter:1930}, we can compute the circumsphere of a simplex with the aid
of the Cayley-Menger matrix.  The Cayley-Menger matrix is given by
\cref{eq:cm}.

\begin{equation} \label{eq:cm}
CM = \begin{bmatrix}
0 & 1 & 1 & 1 & \dots \\
1 & 0 & d_0^1 & d_0^2 & \dots \\
1 & d_1^0 & 0 & d_1^2 & \dots \\
1 & d_2^0 & d_2^1 & 0 & \dots \\
\vdots & \vdots & \vdots & \vdots & \ddots
\end{bmatrix}
\end{equation}

with $d_i^j = \sum_k \left( V_i^k - V_j^k \right)^2$, the squared distance
between vertices $V_i$ and $V_j$.

Then, the radius of the circumsphere is given by \cref{eq:cm-radius}, the
circumcenter by \cref{eq:cm-coords} (please refer to \cite{Westendorp:2013} for
the proof).

\begin{equation} \label{eq:cm-radius}
R = \sqrt{- \frac{1}{2} \cdot {CM^{-1}}_{1,1}}
\end{equation}

\begin{equation} \label{eq:cm-coords}
x_j = {CM^{-1}}_{1,j+1}
\end{equation}

Note that we only need the first row of the inverse $CM$ matrix, not the whole
inverse. Also note that $CM$ is symmetric, which enables simpler decompositions
for the inversion operation. $CM$ is \emph{indefinite} (proof omitted).
Since $CM$ is square symmetric, but not positive-definite, the most efficient
decomposition to perform the inverse, to the best of my knowledge, is the
Bunch–Kaufman decomposition \cite{Bunch:1976}, which is only slightly better
than LUP \cite{wiki:lu-decomposition}.

See also the Cayley-Menger determinant \cite{wiki:cayley-menger-determinant}
for the link between the Cayley-Menger matrix and the volume (or
\emph{content}) of n-simplices.

\section{Method tradeoff}
Starting from a list of vertices coordinates $V_i^j$, which method is the most
efficient computationally? Let's count operations to make sure.

\begin{minipage}{\textwidth} \begin{itemize}
\item Method 1: \begin{itemize}
\item $A$: $n^2$ subtractions
\item $b$: $n(n+1)$ squares, $(n-1)(n+1)$ sums, $n$ subtractions
\item LUP decomposition \cite{wiki:lu-decomposition} and solve.
\end{itemize}
\item Method 2: \begin{itemize}
\item Distance squared for one pair of vertices: $(n-1)$ subtractions, $(n-1)$
      squares, $(n-1)$ sums.
\item Number of unique pairs with non-zero distance: $n(n-1)/2$
\item Bunch-Kaufman decomposition \cite{Bunch:1976} and solve.
\end{itemize} \end{itemize} \end{minipage}

Long story short, the setup for the first method is $\mathcal{O}(n^2)$ in time,
the second method is $\mathcal{O}(n^3)$. Both decompositions are
$\mathcal{O}(n^3)$ with a slight advantage for Bunch-Kaufman. Method 1 is
preferable.
